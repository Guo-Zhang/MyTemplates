\documentclass{beamer}

% theme
\mode<presentation>
{
\usetheme{Warsaw}

\setbeamercovered{transparent}
}

% packges
\usepackage{amsmath}
\usepackage{multirow}
\usepackage{multicol}

% setting
\linespread{1.2}

% title page
\title{Consumer Price Search and Platform Design\\ in Internet Commerce}
\subtitle{Dinerstein et.al. (2014)}
\author{Guo Zhang}
\institute[Universities of]
{
WISE, Xiamen University
}
\date{This Version: \today}
\subject{Literatures}

% body
\begin{document}
\begin{frame}[plain]
\maketitle
\end{frame}

\begin{frame}[plain]
\frametitle{Contents}
\begin{multicols}{2}
% \small
\tableofcontents
\end{multicols}
\end{frame}

\section{Introduction}
\begin{frame}
\frametitle{Background}
\begin{itemize}
\item No definitive study measuring online mark-ups
\item Ways of structuring online search affect price competition and consumer purchase patterns
\end{itemize}
\end{frame}

\begin{frame}
\frametitle{Overview}
\begin{itemize}
\item Note: this subsection is not well-organized. 
\end{itemize}
\end{frame}

\begin{frame}[allowframebreaks]
\frametitle{Literatures Review}
\begin{itemize}
\item Search frictions and price competition
  \begin{itemize}
  \item Theoretical: Stigler (1961)
  \item Empirical: Hortacsu and Syverson (2003); Hong and Shum (2006); Hortacsu et al. (2012)
  \end{itemize}
\item Online Price dispersion
  \begin{itemize} 
  \item Bailey (1998); Smith and Brynjolfsson (2001); Baye, Morgan, and Scholten (2004); Einav et al. (forthcoming)
  \end{itemize}
\item Price elasticity estimation
  \begin{itemize}
  \item Ellison and Ellison (2009); Einav et al. (2014)
  \end{itemize}
\framebreak
\item Limited consumer search
  \begin{itemize}
  \item Malmendier and Lee (2011)
  \end{itemize}
\item Consumer search across different websites
  \begin{itemize}
  \item Ellison and Ellison (2014)
  \end{itemize}
\item Two-sided matching
  \begin{itemize}
  \item Fradkin (2014); Horton (2014)
  \end{itemize}
\end{itemize} 
\end{frame}

\section{Search Design in Online Markets}
\begin{frame}
\frametitle{Two Dimensions of Consumer Online Search}
\begin{itemize}
\item Guide consumers toward relevant products
  \begin{itemize}
  \item User query
  \item Advertising
  \item Recomendations
  \end{itemize}
\item "Price search": Help consumers to find attractive prices (focused by this paper)
\end{itemize}
\end{frame}

\begin{frame}
\frametitle{Different Approaches for Search Problems}
\begin{itemize}
\item Identifying relevant goods: SKUs or catalog numbers % what is SKU?
\item Presenting information
  \begin{itemize}
  \item Ordered by listing date: Craigslist % what's Craigslist?
  \item Ordered by price: Amazon
  \item Between two approaches: Google Shopping
  \end{itemize}
\end{itemize}
\end{frame}

\begin{frame}
\frametitle{Trade-offs of Different Approaches to Search Design}
\begin{itemize}
\item Ordered by listing date
  \begin{itemize}
  \item Advantages: more difficult for buyers to find the lowest prices
  \item Disadvantages: provide opportunities for sellers less professional in categorizing products
  \end{itemize}
\item Ordered by prices
\begin{itemize}
\item Advantages: promote price competition
\item Disadvantages: provide sellers with incentives to "obfuscate"
\end{itemize}
\end{itemize}
\end{frame}

\begin{frame}
\frametitle{Redesign of eBay's Search Processes ??}
% Figure 2
\begin{itemize}
\item Before: Best Match
  \begin{itemize}
  \item Not for individual users
  \item Not consider price explicitly % directly 
  \item Difficulty for filtering unrelated goods
  \end{itemize}
\item After: two-stage design
  \begin{itemize}
  \item Search pages with relevant product models
  \item Product pages with top-rated seller presented together, ranked by the lowest posted price+shipping (but never seen)
  \end{itemize}
\end{itemize}
\end{frame}

\section{Effect of Platform Change on Search and Prices}
\begin{frame}
\frametitle{Data Source}
\begin{itemize}
\item Time horizon 
\begin{itemize}
\item Before: traditional Best Match
\item After: new product page as default
\end{itemize}
\item Category
\begin{itemize}
\item Most commonly transacted (to avoid changes during the sample period with half a year)
\end{itemize}
\end{itemize}
\end{frame}

\begin{frame}
\frametitle{Table 1}
\end{frame}

\begin{frame}
\frametitle{General Patterns}
\begin{itemize}
\item Average number of listings: 16-41
\item Variation in prices (measured by 75th price/25th price)
\item Extreme prices: dramatic
\item Consumer purchase goods with 25-40th price percentile at most
\end{itemize}
\end{frame}

\begin{frame}
\frametitle{Comparison between Two Periods}
\begin{itemize}
\item Variation in decreases of transacted prices 
\item Variation in decreases of post prices (reflected) 
\end{itemize}
\end{frame}

\begin{frame}
\frametitle{Consumer Search Patterns}
\begin{itemize}
\item Consumers buy cheaper items ??
\item Shares of top-rated sellers increase
\item Low-prices top-rated sellers promoted
\end{itemize}
\end{frame}

\begin{frame}
\frametitle{Figure 3}
\end{frame}

\begin{frame}
\frametitle{Consideration Set}
\begin{itemize}
\item Data: Halo Reach
\item Changes:
\begin{itemize}
\item Size increases
\item Clicks decreases
\end{itemize}
\end{itemize}
\end{frame}

\section{Model}
\subsection{Consumer Demand}
\begin{frame}
\frametitle{Utility Function}
$$
u_{ij}  =  \alpha_0+\alpha_1 p_j + \alpha_2 TRS_j+ \alpha_3 p_j TRS_j + \varepsilon_{ij}
$$
\begin{itemize}
\item i: consumer
\item j: product
\item p: price
\item TRS: top-rated seller
\begin{itemize}
\item TRS = 1: $u_{ij}  =  (\alpha_0+\alpha_2)+(\alpha_1+\alpha_3) p_j + \varepsilon_{ij}$
\item TRS = 0: $u_{ij}  =  \alpha_0+\alpha_1 p_j + \varepsilon_{ij}$
\end{itemize}
\item $\varepsilon$: logit error, Type I extreme value
\end{itemize}
\end{frame}

\begin{frame}
\frametitle{Consideration Set}
\begin{itemize}
\item Consumers choose utility-maximizing option in their consideration set $J_i \subseteq \mathbf{J}$
\begin{itemize}
\item Consideration set: $J_i$
\item Set of all available offerings $\mathbf{J}$
\end{itemize}
\item Outside goods: $u_{i0} = \varepsilon_{i0}$
\end{itemize}
\end{frame}

\begin{frame}
\frametitle{Demand Parameter Estimation}
\begin{itemize}
\item Browsing data $\to$ Consideration set and resulting choices ???
\item Assumption: the consideration set includes all the listings on the page seen by the consumer following his last search query
\begin{itemize}
\item Before: listings page
\item After: product page
\end{itemize}
\item Demand estimation: multinomial logit choice probabilities
\end{itemize}
\end{frame}

\subsection{Consideration Set}
\begin{frame}
\frametitle{Sample Weight}
\begin{itemize}
\item Question: which sellers make it into the consideration set?
\item Sample weight\\(Wallenius' non-central hypergeometric distribution): 
$$
w_j = exp[
-\gamma 
(
\frac{p_j- min_{k\in \mathbf{J}} (p_k)}{std_{k\in \mathbf{J}}(p_k)}
)
]
$$
\begin{itemize}
\item Before: $\gamma=0$, price did not factor directly into search ranking
\item After: $\gamma>0$, price plays a predominant role  
\end{itemize}
\end{itemize}
\end{frame}

\begin{frame}
\frametitle{Wallenius' Non-central Hypergeometric Distribution}
\begin{itemize}
\item Hypergeometric distribution: \\ \quad the probability of k successes in n draws without replacement from a finite population of size N that contains exactly K successes
\item Non-central hypergeometric distribution: unequal weight for each success   ??
  \begin{itemize}
  \item Wallenius: Competition between successes
  \item Fisher: Simultaneously or independently of each other
  \end{itemize} 
\end{itemize}
\end{frame}

\subsection{Price Behavior}
\begin{frame}
\frametitle{Nash Equlibrium}
$$
\max_{p_j}(p_j-c_j)D_j(p_j)
$$
\begin{itemize}
\item $D_j(p_j)$: probability of a given buyer selects j's product, given the set of offerings $\mathbf{J}$ ??
\footnotesize{
$$
D_j(p_j) = \sum_{J: j\in J \subseteq \mathbf{J}}
[
\frac{exp( \alpha_0+\alpha_1 p_j + \alpha_2 TRS_j+ \alpha_3 p_j TRS_j)}{1+sum_{k\in J}exp( \alpha_0+\alpha_1 p_k + \alpha_2 TRS_k+ \alpha_3 p_k TRS_k)}
]
Pr(J|\mathbf{J})
$$
}
\normalsize
\item $p_j$: sellers do not change prices often in practice in the short run
\end{itemize}
\end{frame}

\begin{frame}[allowframebreaks]
\frametitle{Price Incentive of Sellers}
\begin{itemize}
\item $D_j(p_j)=A_j(p_j)Q_j(p_j)$
  \begin{itemize}
  \item $A_j$:  probability that the listing enters the consideration set given $p_j$ and $\mathbf{J}$ 
  \item $Q_j$: probability that the consumer purchases item j conditional on being in the listing set
  \end{itemize}
\item Optimal price:
$$
\frac{p_j}{c_j}
=(1+\frac{1}{\eta_D})^{-1}
=(1+\frac{1}{\eta_A+\eta_Q})^{-1}
$$
  \begin{itemize}
  \item $\eta_D$,$\eta_A$,$\eta_Q$: respective price elasticities
  \end{itemize}
\framebreak
\item 
\begin{tabular}{lllll}
\multirow{4}{1cm}{$\gamma>0$} 
& & $\eta_A < 0$ &($\gamma \uparrow \to \eta_A$) \\
& $\nearrow$ & \\
& $\searrow$ & \\
& & $\eta_Q < 0$ \\
\end{tabular}
\end{itemize}
\end{frame}

\subsection{Discussion}
\begin{frame}
\frametitle{ Stahl's (1989) Search Model}
\begin{itemize}
\item Two types of consumers:
\begin{itemize}
\item who (optimally) sample a single offer completely at random
\item who sample all the offers
\end{itemize} 
          $\to$ $L \in \{1,|\mathbf{J}| \}$ and $\gamma=0$  ??
\item ??
\end{itemize}
\end{frame}

\begin{frame}
\frametitle{Directions to Be Extend}
\begin{itemize}
\item Heterogeneity among sellers or consumers
\begin{itemize}
\item Seller: distinguish between price-elastic searchers and price-inelastic "convenience" shoppers ??
\end{itemize}
\item Search rank
\begin{itemize}
\item Less dramatic
\item Harder to interpret
\end{itemize}
\end{itemize}
\end{frame}

\section{Empirical Estimates}
\subsection{Estimation Sample}
\begin{frame}
\frametitle{Estimation Sample}
\begin{itemize}
\item Product: single, well-defined - popular Microsoft Xbox 360 video game, Halo Reach
\begin{itemize}
\item A large number of units transact on eBay
\item Relatively stable supply and demand during the observation period
\end{itemize}
\item 
\end{itemize}
\end{frame}

\subsection{Descriptive Statistics}

\subsection{Model Estimates}
\begin{frame}
\frametitle{Demand Parameters: Methods}
\begin{itemize}
\item Standard logit demand with individual-level data and observed individual-specific consideration sets
\item Maximum likelihood, restricting attention only to consumer data from the \textbf{before} period
\end{itemize}
\end{frame}

\begin{frame}
\frametitle{Demand Parameters: Results}
% Table 3
\begin{itemize}
\item Top-rated sellers(TRS): \$10 discount (of an average price of less than \$40) - very large(no advantage for the before period)
\item Price elasticity: -10 (-13 for TRS)
\item Profit margin(profit/revenue): 10\%
\end{itemize}
\end{frame}

\begin{frame}
\frametitle{Consideration Set Model: Methods}
\begin{itemize}
\item Estimate distribution of $L_i$ (the number of items sampled by a consumer) 
\begin{itemize}
\item directly from the browsing data
\item separately for the before and after periods
\end{itemize}
\item Estimate the sampling parameter $\gamma$
\end{itemize}
\end{frame}

\begin{frame}
\frametitle{Consideration Set Model: Results}
\begin{itemize}
\item Distribution of $L_i$: Figure 3
% Figure 3
\item Sample parameter $\gamma$
  \begin{itemize}
  \item Before: 0 
  \item After : 0.81 \\?? - a ten percent reduction in the posted price would, on average, make the listing 29\% more likely to be part of a consumer's consideration set
  \end{itemize}
\end{itemize}
\end{frame}

\begin{frame}
\frametitle{Seller Costs: Methods}
$$
c_j = p_j +\frac{D_j(p_j)}{D'_j(p_j)}
$$
\begin{itemize}
\item Demand parameters + consideration set model $\to D_j$
\item First order condition: $D_j \to D'_j$
\item Back out the cost $c_j$
\end{itemize}
\end{frame}

\begin{frame}
\frametitle{Mathematical Note: Proof}
\begin{align*}
f(p_j)&=(p_j-c_j)D_j(p_j) \\
\intertext{First order condition:}
\frac{d f}{d p_j} &= D_j(p_j)+ (p_j-c_j)D'_j(p_j)  \\
&= 0 \\
\intertext{Therefore,}
c_j &= p_j +\frac{D_j(p_j)}{D'_j(p_j)} \\
\end{align*}
\end{frame}

\begin{frame}
\frametitle{Seller Costs: Results}
\begin{itemize}
\item Figure 4
\item High dispersion of seller costs
\end{itemize}
\end{frame}

\section{Applying the Model}
\subsection{Changing the Search Design}
\begin{frame}
\frametitle{Methods}
\begin{itemize}
\item Assumption: consumer choice behavior and sell cost distribution remain unchanged
\item Method: combine our demand and cost estimates from the before period with estimates of the consideration set process from the after period
\item Goal: calculate equilibrium prices and expected sales with the post-redesign search process
\end{itemize}
\end{frame}

\begin{frame}
\frametitle{Results}
\begin{itemize}
\item Demand is more responsive to seller prices
  \begin{itemize}
  \item Demand becomes more elastic
  \item Seller margin falls by 20\%
  \end{itemize}
\end{itemize}
\end{frame}

\begin{frame}
\frametitle{Factors Contributed to the Shift of Seller Incentives}
\begin{itemize}
\item Increase in the size of consideration set
\item Price became an important factor in entering the consideration set
\item Increase in the number of available listings
\end{itemize}
\end{frame}

\begin{frame}[allowframebreaks]
\frametitle{Evaluate the Importance Three Factors}
\begin{itemize}
\item Methods
\begin{itemize}
\item Separately impose three effects
\item Calculate new price equilibrium $\to$ equilibrium margins and purchase rates
\end{itemize}
\item Results
  \begin{itemize}
  \item Increase in $\gamma$ - large effect
  \item Increase in listings - small effect
  \item Increase in consideration set size - small effect
  \end{itemize}
\framebreak
\item Question
  \begin{itemize}
  \item Whether the model's predictions for the after period are similar to the outcomes we actually observe
  \end{itemize}
\item Comparsion
  \begin{itemize}
  \item Distribution of seller prices: match quite well (Figure 6)
  \item Consumer purchase rate: reasonably close (Table 3) 
  \end{itemize}
\end{itemize}
\end{frame}

\subsection{Search Friction and Price Dispersion}
\begin{frame}
\frametitle{Reasons for High Degree of Price Dispersion}
\begin{itemize}
\item Dispersion in costs
\item Search frictions
\item Perceived seller differentiation
\end{itemize}
\end{frame}

% Tough math

\subsection{Discussion and Extensions}
\begin{frame}
\frametitle{Discussion and Extensions}
\begin{itemize}
\item 
\end{itemize}
\end{frame}

\section{A/B Experiment}
\begin{frame}
\frametitle{Experiment Design}
\begin{itemize}
\item Method:
\begin{itemize} 
\item Users were randomly assigned to be shown either product page or Best Match results in response to a search query if the product page existed
\item After being shown initial results, users could browse to the other type of listing
\end{itemize}
\item Goal: test whether conditional on both types of results being available, it was better to start users with relevance results
\end{itemize}
\end{frame}

\begin{frame}
\frametitle{Phenomena}
\begin{itemize}
\item The experiment did succeed in steering users toward particular results
\item Best Match group had a higher purchase rate
\item The Best Match group also had \textbf{slightly} higher average transacted prices
\end{itemize}
\end{frame}

\begin{frame}
\frametitle{Data}
\begin{itemize}
\item Observations: All purchases from the experimental user sessions $\to$ Select product pages that were visited at least 1,000 times in the experiment
\item Entities: 4,250 different products, and 30,696 different listings that had purchases
\item Period: July 25, 2012 to August 30, 2012
\end{itemize}
\end{frame}

\begin{frame}
\frametitle{Measuring Product Homogeneity}
\begin{itemize}
\item Relevance ranking might have been particularly effective for differentiated products
\item $\to$ Need a proxy for each product's level of homogeneity
\item $\to$ The fraction of product listings with the most common title on the product code
\item $\to$ Group products depend on whether their top listing share is in the top quartile (less heterogeneous), middle half, or bottom quartile (more heterogeneous) ??
\end{itemize}
\end{frame}

\begin{frame}
\frametitle{Results}
\begin{itemize}
\item The Best Match treatment looks best for the more heterogeneous products
  \begin{itemize}
  \item Purchases under the product page
  \item Average percentage effect on sales
  \end{itemize}
\item  Price search problem is just one dimension of the broader platform problem when there are a large variety of products, many of which are heterogeneous and may involve richer consumer search processes
\end{itemize}
\end{frame}

\section{Conclusion}
\begin{frame}
\frametitle{Conclusion}
\begin{itemize}
\item Explore search frictions in online commerce, and the role of search design in reducing them
\item Price search and price competition and homogenous products
\item Develop from theory literatures 
\item Explain price dispersion, seller margins and the effects of changes in the search ranking
\end{itemize}
\end{frame}

\begin{frame}
\frametitle{Shortcomings}
\begin{itemize}
\item Price just one of the dimensions along which consumers are searching
\item Orienting a platform toward price search may not work as well for heterogeneous products
\end{itemize}
\end{frame}


\begin{frame}[plain]
\frametitle{Contents}
\begin{multicols}{2}
% \small
\tableofcontents
\end{multicols}
\end{frame}

\end{document}