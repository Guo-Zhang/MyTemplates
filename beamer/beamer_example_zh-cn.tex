\documentclass[13pt]{beamer}
% xeLaTeX

% theme
\mode<presentation>
{
\usetheme{Warsaw}

\setbeamercovered{transparent}
}

% packages and settings
\usepackage{ctex}

\usepackage{xcolor}
\definecolor{mygray}{RGB}{245,245,245}

\usepackage{listings}
\lstset{language=Python}
\input{listings-python.prf}
\lstset{escapeinside=``}
\lstset{numbers=left}
\lstset{breaklines}
\lstset{backgroundcolor=\color{mygray}}

\usepackage{multicol}

\usepackage{hyperref}
\hypersetup{urlcolor=blue}

% settings
\setbeamertemplate{frametitle continuation}{}

 \AtBeginSection[]
 {
 \begin{frame}<beamer>[plain]
 \frametitle{目录}
 \begin{multicols}{2}
 \tableofcontents[currentsection,hideothersubsections,currentsubsection]
 \end{multicols}
 \end{frame}
 }
 
 \AtBeginSubsection[]
 {
 \begin{frame}<beamer>[plain]
 \frametitle{目录}
 \begin{multicols}{2}
 \tableofcontents[currentsection,hideothersubsections,currentsubsection]
 \end{multicols}
 \end{frame}
 }


% body
\begin{document}


\title{Python快速入门}
\author{张果}
\institute[Universities of]
{
厦门大学\quad 王亚南经济研究院}
\renewcommand{\today}{\number\year 年 \number\month 月 \number\day 日}
\date{本版本:\today}
\subject{Lectures}


\begin{frame}
\titlepage
\end{frame}

\begin{frame}[plain]
\frametitle{目录}
\begin{multicols}{2}
\tableofcontents
\end{multicols}
\end{frame}

\section{引言}
\begin{frame}
\frametitle{什么是程序?}
\begin{itemize}
  \item 程序:输入 $\to$ 输出
  \item 程序 = 数据结构 + 算法
  \begin{itemize}
    \item 数据结构:储存、组织数据
    \item 算法:具体步骤
  \end{itemize}
  \item 程序:实现需求,解决问题
\end{itemize}
\end{frame}

\begin{frame}
\frametitle{为什么选择Python?}
\begin{itemize}
\item 易学易用,开发效率高
\item 强大的胶水
\end{itemize}
\end{frame}

\begin{frame}
\frametitle{如何选择Python版本?}
\begin{itemize}
\item 2.x vs 3.x?
\begin{itemize}
\item 为什么3.x无法完全取代2.x?没有真正解决2.x的问题。
\item 2.x和3.x互相迁移很困难?不是。
\end{itemize}
\item 2.x:选择2.6以上版本
\item 3.x:选择3.4以上版本
\item 本教程使用2.7.y(y>=9)
\end{itemize}
\end{frame}

\begin{frame}
\frametitle{Python开发环境}
\begin{itemize}
\item \href{https://www.python.org}{\underline{Python官方工具}}
\item \href{http://www.jetbrains.com/pycharm/}{\underline{PyCharm:主流IDE}}
\item \href{https://www.continuum.io/downloads}{\underline{Anaconda:科学计算专用}}
\item \href{http://jupyter.org}{\underline{Jupyter:浏览器版iPython}}
\item \href{https://www.sublimetext.com/3}{\underline{Sublime Text 3:可拓展性极强的文本编辑器}}
\end{itemize}
\end{frame}

\section{基础知识}
\begin{frame}[containsverbatim]
\frametitle{Python解释器}
\begin{lstlisting}
>>> # `输出0到 9`
>>> for i in range(10):
...     print i,
...
0 1 2 3 4 5 6 7 8 9
\end{lstlisting}
\begin{itemize}
  \item \lstinline{>>>}     \#主提示符
  \item \lstinline{...}         \#次提示符
  \item \lstinline{#}         \#注释
\end{itemize}
\end{frame}

\begin{frame}[containsverbatim]
\frametitle{语句与表达式}
\begin{itemize}
  \item 语句:\lstinline{print 'Hello,world'}
  \item 表达式:\lstinline{abs(-4)}
\end{itemize}
\end{frame}

\begin{frame}[containsverbatim]
\frametitle{帮助}
\begin{lstlisting}
>>> help(input)
Help on built-in function input in module __builtin__:

input(...)

    input([prompt]) -> value
    
    Equivalent to eval(raw_input(prompt)).
    
\end{lstlisting}
\end{frame}

\begin{frame}[containsverbatim]
\frametitle{帮助}
\begin{lstlisting}
>>> help(print)
  File "<stdin>", line 1
    help(print)
             ^
SyntaxError: invalid syntax
\end{lstlisting}
\end{frame}

\begin{frame}[containsverbatim]
\frametitle{操作符}
\begin{itemize}
\item 算术操作符:+,-,*,/,//,\%,**
\begin{lstlisting}
>>> 1/2,1//2,1.0/2,1.0//2
(0, 0, 0.5, 0.0)
>>> 4**2
16
>>> 7%3
1
\end{lstlisting}
\item 比较操作符:<,<=,>,>=,==,!=
\item 逻辑操作符:\lstinline{and,or,not}
\end{itemize}
\end{frame}

\begin{frame}[containsverbatim]
\frametitle{变量与赋值}
\begin{lstlisting}
>>> counter = 0
>>> # `给变量counter赋值0`
\end{lstlisting}
\end{frame}

\section{输出与输入}
\subsection{输出}
\begin{frame}[containsverbatim]
\frametitle{输出}
\begin{itemize}
  \item 交互式解释器
\begin{lstlisting}
>>> 'Hello,world'
'Hello,world'
>>> _  # `最后一个表达式的值`
'Hello,world'
\end{lstlisting}
  \item print语句
\begin{lstlisting}
>>> print 'Hello,world'
Hello,world
\end{lstlisting}
\end{itemize}
\end{frame}

\begin{frame}
\frametitle{输出}
注释:
\begin{itemize}
\item \lstinline{print}调用Python对象的\lstinline{__str__()}方法;
\item 交互式解释器调用\lstinline{__repr__()}方法。
\end{itemize}
\end{frame}

\begin{frame}[containsverbatim]
\frametitle{格式化输出}
\begin{lstlisting}
>>> print "%d, my favorate %s, but %f" %(1,'Python',1.1)
1, my favorate Python, but 1.100000
\end{lstlisting}
\begin{itemize}
  \item \%d: 整型
  \item \%f: 浮点型
  \item \%s: 字符串
\end{itemize}
\end{frame}

\begin{frame}[containsverbatim]
\frametitle{重定向到日志}
\begin{lstlisting}
log = open('C:/test.log','a')
print >> log,'My favorate food'
print >> log,'My favorate drink'
log.close()
\end{lstlisting}
\end{frame}

\subsection{输入}
\begin{frame}[containsverbatim]
\frametitle{input}
\begin{lstlisting}
>>> input('please input your favorate number:')
please input your favorate number:7
7
>>> type(_)
<type 'int'>
\end{lstlisting}
\end{frame}

\begin{frame}[containsverbatim]
\frametitle{raw\_input}
\begin{lstlisting}
>>> raw_input('please input your favorate number:')
please input your favorate number:7
'7'
>>> type(_)
<type 'str'>
\end{lstlisting}
\end{frame}

\section{基本数据类型}
\subsection{数字}
\begin{frame}[containsverbatim]
\frametitle{数字}
\begin{itemize}
\item 整型
\begin{itemize}
  \item 有符号整型(\lstinline{int}): \lstinline{1}
  \item 长整型(\lstinline{long}): \lstinline{100000L,100000l}
  \item 布尔值(\lstinline{bool}): \lstinline{True, False}
\end{itemize}
\item 浮点型(\lstinline{float}): \lstinline{1.1}
\item 复数(\lstinline{complex}): \lstinline{1+1j,1+1J}
\end{itemize}
\end{frame}

\begin{frame}[containsverbatim,allowframebreaks]
\frametitle{数字}
\begin{lstlisting}
>>> type(1)
<type 'int'>
>>> type(100000L)
<type 'long'>
>>> type(True)
<type 'bool'>
>>> type(1.1)
<type 'float'>
>>> type(1+1j)
<type 'complex'>
\end{lstlisting}
\end{frame}

\subsection{序列}
\begin{frame}[containsverbatim]
\frametitle{序列(senquence)}
\begin{itemize}
\item 列表(\lstinline{list}): [1,3,4]
\item 元组(\lstinline{tuple}): (1,3,4,5)
\item 字符串(\lstinline{str}): 'Hello,world',"Hello,world"
\end{itemize}
\end{frame}

\begin{frame}[containsverbatim]
\frametitle{序列(senquence)}
\begin{lstlisting}
>>> type([1,3,4])
<type 'list'>
>>> type((1,3,4,5))
<type 'tuple'>
>>> type('hello,world')
<type 'str'>
\end{lstlisting}
\end{frame}

\begin{frame}[containsverbatim,allowframebreaks]
\frametitle{序列(senquence)}
\begin{lstlisting}
>>> # `创建序列`
>>> aList = [1,2,3,4]
>>> aTuple = (1,2,3,4)
>>> aStr = 'hello,world'
>>> # `索引`
>>> aList[0]
1
>>> aTuple[3]
4
>>> aStr[4]
'o'
>>> # `切片`
>>> aList[0:2]
[1, 2]
>>> aTuple[:3]
(1, 2, 3)
>>> aStr[4:]
'o,world'
\end{lstlisting}
\end{frame}

\subsection{集合和映射类型}
\begin{frame}[containsverbatim]
\frametitle{集合和映射类型}
\begin{itemize}
\item 集合(\lstinline{set}): \{1,5,6\}
\item 字典(\lstinline{dict}): \{'xiaoming':1,'xiaohong':2\}
\end{itemize}
\end{frame}

\begin{frame}[containsverbatim]
\frametitle{集合和映射类型}
\begin{lstlisting}
>>> type({1,5,6})
<type 'set'>
>>> type({'xiaoming':1,'xiaohong':2})
<type 'dict'>
\end{lstlisting}
\end{frame}

\begin{frame}[containsverbatim,allowframebreaks]
\frametitle{字典}
\begin{lstlisting}
>>> aDict = {'x':1}  # `创建字典`
>>> aDict['y']=2  # `添加键值对`
>>> aDict
{'y': 2, 'x': 1}
>>> aDict['x']  # `索引字典`
1
\end{lstlisting}
\framebreak
\begin{lstlisting}
>>> aDict.keys()  # `字典的键`
['y', 'x']
>>> aDict.values()  # `字典的值`
[2, 1]
>>> aDict.items()  # `键值对`
[('y', 2), ('x', 1)]
\end{lstlisting}
\end{frame}

\section{基本控制语句}
\subsection{if语句}
\begin{frame}[containsverbatim]
\frametitle{if}
\begin{lstlisting}
>>> a = 4
>>> if a<10:
...     print 'Hello,world'
...
Hello,world
\end{lstlisting}
注释:代码块\textbf{缩进}表达逻辑(一般是四个空格)
\end{frame}

\begin{frame}[containsverbatim]
\frametitle{if-else}
\begin{lstlisting}
>>> a = 4
>>> if a<10:
...     print 'Hello,world'
... else:
...     print 'Hello'
...
Hello,world
\end{lstlisting}
\end{frame}

\begin{frame}[containsverbatim]
\frametitle{if-elif-else}
\begin{lstlisting}
>>> a = 4
>>> if a<10:
...     print 'Hello,world'
... elif a<15:
...     print 'Hello?'
... else:
...     print 'Hello'
...
Hello,world
\end{lstlisting}
\end{frame}


\subsection{for语句}
\begin{frame}[containsverbatim]
\frametitle{迭代器循环}
\begin{lstlisting}
>>> for item in {'my','your','their','his','her'}:
...     print item,
...
their his my your her
\end{lstlisting}
注释:list,tuple,set,str等都是可迭代对象
\end{frame}

\begin{frame}[containsverbatim]
\frametitle{用range实现计数器循环}
\begin{lstlisting}
>>> for i in range(10):
...     print i,
...
0 1 2 3 4 5 6 7 8 9
\end{lstlisting}
\end{frame}

\begin{frame}[containsverbatim]
\frametitle{enumerate}
\begin{lstlisting}
>>> foo = 'abcdefglijklmn'
>>> for i,ch in enumerate(foo):
...     print ch,'(%d)'%i,',',
...
a (0) , b (1) , c (2) , d (3) , e (4) , f (5) , g (6) , l (7) , i (8) , j (9) , k (10) , l (11) , m (12) , n (13) ,
\end{lstlisting}
\end{frame}

\begin{frame}[containsverbatim]
\frametitle{zip}
\begin{lstlisting}
>>> a = [1,2,3]
>>> b = (2,3,4)
>>> for i,j in zip(a,b):
...     print i,j
...
1 2
2 3
3 4
\end{lstlisting}
\end{frame}

\begin{frame}[containsverbatim]
\frametitle{列表解析}
\begin{lstlisting}
>>> [x**2 for x in range(4)]
[0, 1, 4, 9]
>>> [x**2 for x in range(10) if ((x**2)%2)]
[1, 9, 25, 49, 81]
\end{lstlisting}
\end{frame}

\begin{frame}[containsverbatim]
\frametitle{字典解析}
\begin{lstlisting}
>>> {x:x**2 for x in range(4)}
{0: 0, 1: 1, 2: 4, 3: 9}
\end{lstlisting}
\end{frame}

\begin{frame}[containsverbatim]
\frametitle{集合解析}
\begin{lstlisting}
>>> {x**2 for x in range(4)}
set([0, 1, 4, 9])
\end{lstlisting}
\end{frame}

\begin{frame}[containsverbatim]
\frametitle{生成器(generator)}
\begin{lstlisting}
>>> (x**2 for x in range(4))
<generator object <genexpr> at 0x0000000002D25678>
>>> for i in _:
...    print i,
...
0 1 4 9
\end{lstlisting}
\end{frame}

\subsection{while语句}
\begin{frame}[containsverbatim]
\frametitle{计数器循环}
\begin{lstlisting}
>>> a = 0
>>> while a<10:
...    print a,
...    a = a + 1
\end{lstlisting}
\end{frame}

\begin{frame}[containsverbatim]
\frametitle{无限循环}
\begin{lstlisting}
>>> a = 0
>>> while True:
...    print a,
...    a = a + 1
\end{lstlisting}
注意:慎用。一般在服务器程序等类似情况下中使用。
\end{frame}

\subsection{break和continue}
\begin{frame}[containsverbatim,allowframebreaks]
\frametitle{break和continue}
\begin{itemize}
\item \lstinline{break}: 结束当前循环,跳转到下个语句
\item \lstinline{continue};终止当前循环,忽略剩余语句,回到循环顶端
\end{itemize}
\begin{lstlisting}
>>> a = 0
>>> while a<100:
...     a = a + 1
...     if not a%5:
...         continue
...     print a,
...     if a == 97:
...          break
...
1 2 3 4 6 7 8 9 11 12 13 14 16 17 18 19 21 22 23 24 26 27 28 29 31 32 33 34 36
37 38 39 41 42 43 44 46 47 48 49 51 52 53 54 56 57 58 59 61 62 63 64 66 67 68 69 
71 72 73 74 76 77 78 79 81 82 83 84 86 87 88 89 91 92 93 94 96 97
\end{lstlisting}
\end{frame}

\subsection{pass语句}
\begin{frame}[containsverbatim]
\frametitle{pass语句}
\begin{itemize}
\item 功能:不做任何事情
\begin{itemize}
\item 开发和调试
\item 异常处理
\end{itemize}
\end{itemize}
\begin{lstlisting}
>>> for i in range(10):
...     pass
...
>>>
\end{lstlisting}
\end{frame}

\section{文件和文件系统操作}

\subsection{文件}
\begin{frame}[containsverbatim]
\frametitle{写入文件}
\begin{lstlisting}
>>> import os
>>> os.getcwd()  # get current working dictionary
>>> f = open('test.txt','w')
>>> f.write('Hello,world')
>>> f.close()
\end{lstlisting}
\end{frame}

\begin{frame}[containsverbatim]
\frametitle{读取文件}
\begin{lstlisting}
>> import os
>> os.getcwd()
>> f = open('test.txt','r')
>> f.read()
>> f.close()
\end{lstlisting} 
\end{frame}

\subsection{文件系统}
\begin{frame}
\frametitle{os模块:文件处理}
\begin{itemize}
  \item \lstinline{os.remove(fname)}  \# 删除文件
  \item \lstinline{os.rename(oldname,newname)}  \# 重命名/移动文件
  \item \lstinline{os.walk(path)} \# 生成一个目录树下所有文件
\end{itemize}
\end{frame}

\begin{frame}
\frametitle{os模块:目录和文件夹}
\begin{itemize}
  \item \lstinline{os.chdir(path)}  \# 改变当前工作目录
  \item \lstinline{os.listdir(path)}  \# 列出指定目录的文件名
  \item \lstinline{os.getcwd()} \# 返回当前工作目录
  \item \lstinline{os.mkdir(path)} \# 创建目录
\end{itemize}
\end{frame}

\begin{frame}
\frametitle{os.path模块:路径名分隔}
\begin{itemize}
  \item \lstinline{os.path.basename(fname)} \# 去掉目录路径,返回文件名
  \item \lstinline{os.path.dirname(fname)}  \# 去掉文件名,返回目录路径
  \item \lstinline{os.path.join(dirname,basename)} \# 组合目录名和文件名
  \item \lstinline{os.path.split(fname)} \# 拆分目录名和文件名
  \item \lstinline{os.path.splitext(path)} \#拆分路径名和扩展名
\end{itemize}
\end{frame}

\begin{frame}
\frametitle{os.path模块:路径名信息}
\begin{itemize}
  \item \lstinline{os.path.exists(path)} \# 指定路径(文件or目录)
  \item \lstinline{os.path.isfile(fname)}  指定路径是否存在且为一个文件
  \item \lstinline{os.path.isdir(dirname)}  指定路径是否存在且为一个目录
\end{itemize}
\end{frame}

\section{错误与异常}
\begin{frame}[containsverbatim]
\frametitle{try-except语句}
\begin{lstlisting}
>>> try:
...     open('test.txt')
... except IOError:
...     print 'No such file'
...
No such file
\end{lstlisting}
\end{frame}
 
\section{函数}
\subsection{调用函数}
\begin{frame}
\frametitle{调用函数}
\begin{itemize}
  \item 帮助:dir, help
  \item 工厂函数:int, float, str, dict, tuple, list, set, file
  \item 长度:len
  \item 类型:type
  \item 排序:sort, reserved
  \item 数学:max, min, abs
\end{itemize}
\end{frame}

\subsection{定义函数}
\begin{frame}[containsverbatim]
\frametitle{定义函数}
\begin{lstlisting}
>>> def test(arg):
...     return arg*2-4
...
>>> test(4)
4
\end{lstlisting}
\end{frame}

\subsection{函数的参数*}
\begin{frame}
\frametitle{函数的参数*}
\begin{itemize}
  \item 默认参数
  \item 可变参数
  \item 关键词参数
\end{itemize}
参考资料:\href{http://www.liaoxuefeng.com/wiki/001374738125095c955c1e6d8bb493182103fac9270762a000/001374738449338c8a122a7f2e047899fc162f4a7205ea3000}
{\underline{廖雪峰Python2.7教程:函数的参数}}
\end{frame}
 
\section{类*}
\subsection{定义类*}
\begin{frame}[containsverbatim]
\frametitle{定义类*}
\begin{lstlisting}
>>> class Test(object):
...     def __init__(self,myname):
...          self.name = myname
...     def __str__(self):
...          return self.name*2
...     def __repr__(self):
...          return self.name*3
...
>>>
\end{lstlisting}
\end{frame}

\subsection{创建类实例*}
\frametitle{创建类实例*}
\begin{frame}[containsverbatim]
\begin{lstlisting}
>>> Test('Bob')
BobBobBob
>>> print Test('Bob')
BobBob
\end{lstlisting}
\end{frame}

\section{模块}
\subsection{导入和使用模块}
\begin{frame}[containsverbatim,allowframebreaks]
\frametitle{导入和使用模块}
\begin{lstlisting}
>>> import string
>>> dir(string)
['Formatter', 'Template', '_TemplateMetaclass', '__builtins__', '__doc__', '__fi
le__', '__name__', '__package__', '_float', '_idmap', '_idmapL', '_int', '_long'
, '_multimap', '_re', 'ascii_letters', 'ascii_lowercase', 'ascii_uppercase', 'at
of', 'atof_error', 'atoi', 'atoi_error', 'atol', 'atol_error', 'capitalize', 'ca
pwords', 'center', 'count', 'digits', 'expandtabs', 'find', 'hexdigits', 'index'
, 'index_error', 'join', 'joinfields', 'letters', 'ljust', 'lower', 'lowercase',
 'lstrip', 'maketrans', 'octdigits', 'printable', 'punctuation', 'replace', 'rfi
nd', 'rindex', 'rjust', 'rsplit', 'rstrip', 'split', 'splitfields', 'strip', 'sw
apcase', 'translate', 'upper', 'uppercase', 'whitespace', 'zfill']
>>> string.uppercase
'ABCDEFGHIJKLMNOPQRSTUVWXYZ'
>>> string.lowercase
'abcdefghijklmnopqrstuvwxyz'
>>> string.whitespace
'\t\n\x0b\x0c\r '
>>> string.upper
<function upper at 0x0000000002BB8278>
>>> string.upper('agesrag')
'AGESRAG'
\end{lstlisting}
\end{frame}

\subsection{下载第三方模块}
\begin{frame}[containsverbatim]
\frametitle{下载第三方模块}
\begin{itemize}
\item 在命令行输入:
\begin{itemize}
  \item \lstinline{pip install <package>}
  \item \lstinline{easy_insall <package>}
\end{itemize}
\item 常见问题:
\begin{itemize}
  \item
  \href{http://www.crifan.com/run_pip_install_django_error_pip_is_not_recognized_as_an_internal_or_external_command_operable_program_or_batch_file/}
  {\underline{pip不能使用}}
  \item
  \href{https://zhuanlan.zhihu.com/p/21380755?refer=xmucpp}{\underline{C扩展库不能安装}}
\end{itemize}
\end{itemize}
\end{frame}

\begin{frame}[plain]
\frametitle{目录}
\begin{multicols}{2}
\tableofcontents
\end{multicols}
\end{frame}

\end{document}
